\documentclass{article}
\usepackage[utf8]{inputenc}
\usepackage{graphicx}
\usepackage{subcaption}
\usepackage[bottom=0.5in,top=0.5in,right=1in,left=1in]{geometry}

\title{Experiment 3}
\author{Nitish Kumar Thakur \\ RE000144 }
\date{17 August, 2021}

\begin{document}

\maketitle

\section{Objective}
1.Study the AM-DSB full carrier modulation and demodulation and implement  practically using IC-AD633\par2.Study the AM-DSB supressed carrier modulation and demodulation and implement circuit practically using IC-AD633

\section{Components and Equipment Required}                    
 IC AD-633 *Capacitor(0.1uf)-2 *Power supply *Fuction generator *connecting wire *Breadboard *DSO
 
 
 
\section{Theory}
Modulation:-one of parameter amplitude,phase,frequency of carrier signal varied according to message signal amplitude variation


\subsubsection{Double-sideband full carrier Modulation}
In Double side band Full carrier modulation technique amplitude of carrier signal varied with respect to message signal amplitude variation.we uses square wave modulation technique to generate modulated signal
\begin{figure*}[ht]
	\centering
	\includegraphics[scale=1.1]{download.png}
	\caption{DSB-FC}
	\label{DSF}
\end{figure*}

\subsection{Double-sideband full carrier demodulation}
For demodulation of double side band full carrier we uses Envelope detector.proper choice of RC for envelope detection is  \par
1/fc \ll RC \ll 1/fm.\par
for good detection of modulated signal RC must lie between those two value.

\begin{figure*}[ht]
	\centering
	\includegraphics[scale=1.1]{ed.png}
	\caption{Envelope detector}
	\label{ed}
\end{figure*}

\subsection{Double-sideband suppressed carrier Modulation}
In double side band supressed carrier demodulation we just multiply the message signal with carrier signal for this we uses multiplier IC AD-633.
\begin{figure*}[ht]
	\centering
	\includegraphics[scale=0.7]{dsbsc.png}
	\caption{DSB-SC Modulation}
	\label{dsbsc}
\end{figure*}

\subsection{Double-sideband suppressed carrier demodulation}
For demodulation of DSB-SC we use coherent demodulation method.

\begin{figure*}[ht]
	\centering
	\includegraphics[scale=0.5]{dsb.png}
	\caption{DSB-SC Demodulation}
	\label{dsbsc}
\end{figure*}




\subsubsection{IC AD-633}
The AD633 is a four-quadrant, analog multiplier. It includes high impedance, differential X and Y inputs,and a high impedance summing input (Z). 
total accuracy of 2 percent of full scale.\par 100 µV rmsin.........bandwidth 10 Hz to 10 kHz . \par A 1 MHz bandwidth,20 V/µs slew rate \par
The block diagram of IC AD-633 is shown in Figure \ref{BD} 
\begin{figure*}[ht]
	\centering
	\includegraphics[scale=0.5]{IC_block_diagram.JPG}
	\caption{Block diagram of IC AD-633}
	\label{BD}
\end{figure*}


Pin description of IC AD-633 in figure \ref{BDF} 
\begin{figure*}[ht]
	\centering
	\includegraphics[scale=0.5]{IC_block_diagram_description.jpg}
	\caption{Description of different pin of IC AD-633}
	\label{BDF}
\end{figure*}
 
 
The functional diagram of IC AD-633 as a multiplier is shown in Figure \ref{FD}
\begin{figure*}[ht]
	\centering
	\includegraphics[width=1\linewidth]{IC_functional_diagram.JPG}
	\caption{Functional diagram of IC AD-633 as a multiplier}
	\label{FD}
\end{figure*}









\section{Observation/Results}
\subsection{DSB-FC Modulation}
In IC AD-633 we gave message signal in pin 1(+) and pin 2(-).and in Pin 3(+) and 4(-) we gave carrier signal\par
The  message signal taken from DSO we take sin wave the value of signal is 7 Vp-p and frequency-103Hz\par
The value of carrier signal(sin wave) we taken from fuction generator with 2 Vp-p and frequency- 3 KHz.\par
in experiment we supply power +15v and -15v at pin 8 and 5 respectively and we connect o.1uf capacitor at pin 8 and 5.\par
pin-6 connected to pin-3\par
\begin{figure*}[ht]
	\centering
	\includegraphics[scale=0.7]{abc.png}
	\caption{Waveform on DSO for DSB-FC modulated signal}
	\label{dsbsc}
\end{figure*}


\subsection{DSB-FC Demodulation}
for envelope detection codition is 1/fc \ll RC \ll 1/fm \par 
1/3Khz  \ll RC  \ll 1/3Khz \par
0.33msec  \ll RC  \ll 9.708 msec \par
c=0.01uf\par
R=12 kohm
\begin{figure*}[ht]
	\centering
	\includegraphics[scale=0.5]{xyz.png}
	\caption{Waveform after demodulation on DSO}
	\label{dsbsc}
\end{figure*}



\subsection{DSB-FC}
everything just same as DSB-FC just connect pin-6 to ground.
\begin{figure*}[ht]
	\centering
	\includegraphics[scale=0.5]{mn.png}
	\caption{Waveform of DSB-SC after modulation on DSO}
	\label{dsbsc}
\end{figure*}
 

\section{Conclusion}
 in this experiment we see Modulation demodulation technique of Double sideband full/suppressed carrier signal.for this experiment we uses multiplier IC AD-AD633.
\end{document}
