\documentclass{article}
\usepackage[utf8]{inputenc}
\usepackage{graphicx}
\usepackage{subcaption}
\usepackage[bottom=0.5in,top=0.5in,right=1in,left=1in]{geometry}
\usepackage{float}

\title{Experiment-2}
\author{Nitish Kumar Thakur \\ RE000144 }
\date{10 August, 2021}

\begin{document}

\maketitle

\section{Objective}
Want to understand how sampling work.for this we design sample and hold circuit using IC LF-398,and we reconstruct the message signal with The help of Butterworth low pass filter. 

\section{Components and Equipment Required}                
*IC LF-398 *Capacitor(0.01uf)-2 (0.1uf)-2 *Resistor(3.3kohm)-2 *Regulated Power supply *Function generator *Dso *wires/probes
 
 
 
 
\section{Theory}
We want to convert Analog signal to Digital signal because Digital signal generally advantageous over Analog signal.\par
Sampling Theorem:-so we discretize the analog signal so my main aim is how much we take sample from continuous signal so in future if we want to reconstruct the same signal so we are able ,for this there is one theorem  Nyquist theorem, Nyquist theorem tell us if we take minimum sampling rate fs must be greater then 2 times of highest frequency contain in message signal,then we are able to reconstruct the message signal. fs less then 2fm then aliasing will happen.\par
Also fs not equal to 2fm because we are not able to build perfect/ideal reconstruction filter(low pass filter) \par
There is 3 ways to do sampling 

\subsection{Ideal sampling}
In ideal sampling we just multiply Analog signal with impulse train shown in below figure.but problem is we can't generate ideal impulse train practically so this method just for theoretical understanding.
in figure\ref{ideal} we take m(t) and multiply with impulse train of time period Ts.\par

 
\begin{figure}[h]
	\centering
	\includegraphics[scale=0.5]{ideal.JPG}
	\caption{Ideal sampling}
	\label{ideal}
\end{figure}


\subsection{Natural Sampling}
Natural sampling similar to ideal sampling but in Natural sampling we multiply the Message signal with pulse train instead of impulse train.

This things shown in below picture.
\begin{figure*}[h]
	\centering
	\includegraphics[scale=1]{Natural.JPG}
	\caption{Natural sampling}
	\label{Natural}
\end{figure*}


\subsection{Flat top sampling}
The problem with Natural sampling is information stored in the form magnitude. and we know that Magnitude is much more effected by Noise compare to other parameter(phase/frequency),so in flat top sampling
we perform sampling in such way that magnitude just flat even message signal amplitude changes.
This things shown in below picture. 
\begin{figure*}[h]
	\centering
	\includegraphics[scale=0.7]{Flat.JPG}
	\caption{Flat Top sampling}
	\label{Flat}
\end{figure*}
\subsection{Butterworth low pass filter}
  For reconstuction of signal we uses low pass filter that means 2nd order Butterworth low pass filter.here we take R1=R2 and C1=C2 for simplicity
  
\begin{figure*}[ht]
	\centering
	\includegraphics[scale=1]{butter.png}
	\caption{ Butterworth filter}
	\label{butt}
\end{figure*}
  
\subsection{IC-398 and Pin description}

 
\begin{figure*}[ht]
	\centering
	\includegraphics[scale=1]{ic.png}
	\caption{ figure of IC-398}
	\label{pin}
\end{figure*}


\begin{figure*}[ht]
	\centering
	\includegraphics[scale=1]{piin.png}
	\caption{Function of different pin}
	\label{piin}
\end{figure*}

\subsection{Functional Block diagram}
The functional diagram of IC-398 as a multiplier is shown in Figure \ref{FBD}
\begin{figure*}[h]
	\centering
	\includegraphics[scale=0.65]{Functional_Block_Diagr_IC.JPG}
	\caption{Functional Block Diagram of IC-398}
	\label{FBD}
\end{figure*}



\section{Observation and Result}
*pin1-(+15v) and pin4-(-15v)*pin6-0.01uf capacitor*pin7-Ground \par *Pin3-sine wave (2 Vp-p,450Hz )\par
*pin8-square wave(10 Vp-p,4.5Khz).\par
\subsection{Sampling and holding}
in this experiment we uses sample and hold circuit in which two operation were perform 1st sample 2nd holding the signal.so how we do this task? we perform this task through IC LF-398. this IC compare the  the input(message signal) with logic (i.e square wave AND also we take 4.5 khz frequency due to Nyquist sampling theorem) if logic value higher then input(message)signal then it perform sampling if logic is less then signal then it perform hold operation.
here in figure we observe \par


\begin{figure*}[h]
	\centering
	\includegraphics[scale=0.5]{o1.png}
	\caption{Sampling and holding task perform }
	\label{o1}
\end{figure*}
\subsection{Reconstruction}
After sampling my final aim to reconstruct the message signal as it.for this we uses reconstruction filter/2nd order Butterworth low pass filter.in which we gave input from pin5 and take output and see waveform on DSO.the figure looks like 

\begin{figure*}[h]
	\centering
	\includegraphics[scale=0.5]{reconstructio.png}
	\caption{Reconstruction of sampled signal}
	\label{obs}
\end{figure*}






\section{Sources of error}
we can't reconstruct same signal there is always some difference between original message signal and reconstructed signal.
 also if we take sampling value less then 2fm(900hz) impossible to get original signal because aliasing will happen, even if we take sampling frequency=900Hz we are not able to reconstruct the signal because it is impossible to build ideal reconstruction filter.\par
*in reconstruction (figure-9) if we observe there is spike because we also not able to generate perfect square wave due to Gibbs phenomena.
 


\section{Conclusion}
 we perform sampling by sample and hold circuit.and reconstruct the sampled signal through Butterworth low pass filter

\end{document}
