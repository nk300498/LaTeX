\documentclass{article}
\usepackage[utf8]{inputenc}
\usepackage{graphicx}
\usepackage{subcaption}
\usepackage[bottom=0.5in,top=0.5in,right=1in,left=1in]{geometry}
\usepackage[section]{placeins}
\usepackage{float}
\title{Experiment 6}
\author{Nitish Kumar Thakur \\ 21531010 }
\date{7 september, 2021}

\begin{document}

\maketitle

\section{Objective}
Design and fabricate PWM and PPM Modulators using 555 timers.
 
 \section{Components and Equipment Required}                    
 IC-555 *DSO *Power supply (variable)*connecting wire *Breadboard *probes *Resistance *DSO *Function generator
 
\section{Theory}
\subsection{Pulse Width Modulation}
If the duration of pulse or pulse width is changed as per the samples of message signal the resulting modulation is known as pulse width modulation.\par
PWM generator can be design by using a sawtooth generator and comparator.\par
The sawtooth generator generates a sawtooth signal which is used as a sampling signal.\par
The comparator compare the amplitude of modulating signal and the amplitude of sampling signal i.e sawtooth signal.\par
The output of comparator is high as long as amplitude of message signal is greater than that of the sawtooth signal.\par
Thus the duration for which the comparator output remains high is directly proportional to the amplitude of the modulating signal.\par
As result the comparator output is a PWM signal shown in below figure.
\begin{figure}[h]
  \begin{subfigure}[b]{0.5\textwidth}
    \includegraphics[width=\textwidth]{1pwm.png}
    \caption{PWM Modulation}
    \label{fig:1}
  \end{subfigure}
  %
  \begin{subfigure}[b]{0.4\textwidth}
    \includegraphics[width=\textwidth]{pwm.jpg}
    \caption{When we take sawtooth signal}
    \label{fig:2}
  \end{subfigure}
\end{figure}


\subsection{Pulse Position Modulation}
In pulse position modulation scheme in which amplitude and width of the pulses are kept constant ,while the position of each pulse,with reference to the position of reference pulse is changes.\par
In PPM the amplitude and width of pulses are constant therefore transmitter handles constant power output,this is advantage over PWM.




\begin{figure}[h]
  \begin{subfigure}[b]{0.5\textwidth}
    \includegraphics[width=\textwidth]{2ppm.png}
    \caption{PPM Modulation}
    \label{fig:1}
  \end{subfigure}
  %
  \begin{subfigure}[b]{0.4\textwidth}
    \includegraphics[width=\textwidth]{ppm.jpg}
    \caption{}
    \label{fig:2}
  \end{subfigure}
  %
  \begin{subfigure}[b]{0.4\textwidth}
    \includegraphics[width=\textwidth]{ppm1.jpg}
    \caption{}
    \label{fig:2}
  \end{subfigure}
\end{figure}

\subsection{555 timer IC}
This is Analog ic which is used in this Experiment 
555 timer ic generally operated in Three mode\par
1.Astable Multivibrator\par
2.Monostable multivibrator\par
3.Bistable Multivibrators\par



\begin{figure}[h]
  \begin{subfigure}[b]{0.45\textwidth}
    \includegraphics[width=\textwidth]{Block diag.png}
    \caption{Block diagram of 555}
    \label{fig:1}
  \end{subfigure}
  %
  \begin{subfigure}[b]{0.45\textwidth}
    \includegraphics[width=\textwidth]{100000.png}
    \caption{Pin diagram of 555}
    \label{fig:2}
  \end{subfigure}
\end{figure}




\subsubsection{Astable Multivibrator}
 In A-stable configuration the timer continuously changes its output high to low and vice vrsa, which is achieved by charging and discharging of the capacitor connected to pin THR and TRIG of the timer. When output is high the capacitor is charging because the transistor through which discharging has to occur is off, when the capacitor is charged to Vcc (2/3) the output of the comparator connected to reset goes high and  the output of 555 timer goes low and the transistor is turned on ,due to which the capacitor starts discharging and when Vcc goes below Vcc(1/3) the output again goes high and the capacitor again stars charging  along with it. This process continues and we a square wave at the output of the timer.
 Th=0.69(Ra+Rb)c\par
 Tl=0.69Rb*c
 Duty cycle=Ra+Rb/Ra+2Rb
 
 
\begin{figure*}[h]
	\centering
	\includegraphics[scale=0.5]{108 asta.png}
	\caption{Astable Mode}
	\label{FBD}
\end{figure*}


\subsubsection{Monostable multivibrator}
This circuit diagram shows how a 555 timer IC is configured to function as a basic mono-stable multi-vibrator. A mono-stable multi-vibrator is a timing circuit that changes state once triggered, but returns to its original state after a certain time delay.  It  got its name from the fact that only one of its output states is stable.  It is also known as a 'one-shot'.
   
In this circuit, a negative pulse applied at pin 2 triggers an internal flip-flop that turns off pin 7's discharge transistor, allowing C1 to charge up through R1. At the same time, the flip-flop brings the output (pin 3) level to 'high'.  When capacitor C1 as charged up to about 2/3 Vcc, the flip-flop is triggered once again, this time making the pin 3 output 'low' and turning on pin 7's discharge transistor, which discharges C1 to ground. This circuit, in effect, produces a pulse at pin 3 whose width t is just the product of R1 and C1, i.e., t=R1C1
\begin{figure*}[h]
	\centering
	\includegraphics[scale=0.5]{107 mono.png}
	\caption{Monostable Mode}
	\label{FBD}
\end{figure*}

\subsection{Circuit for PWM/PPM Modulation}
In the circuit there are two stages of 555 timers. One is acting as MONO-STABLE MULTI-VIBRATOR while the second one is acting as an ASTABLE MULTI-VIBRATOR.The input signal is applied to a-stable stage and Its out-put is connected to the input of mono-stable stage which at its output gives us, pulse position modulated signal. The input signal is applied to the control terminal of the a-stable multi-vibrator when the input signal amplitude varies the control terminal voltage changes and the switching voltage of the a-stable multi-vibrator changes accordingly and the width of the output pulse of the a-stable multi-vibrator changes and the time when the mono-stable is triggered varies with the width of the output of the Astable multi-vibrator and the position of the output pulses

\begin{figure*}[h]
	\centering
	\includegraphics[scale=1]{ppmpwm ckt.jpg}
	\caption{Generation of PWM/PPM}
	\label{FBD}
\end{figure*}

















\section{Observation/Results}
We take \par R1=10Kohm \par R2=10Kohm \par C=0.01uf \par VDD-5V.

\subsection{when 555 timer operated in astable mode}
in this we did not give message signal this is simple square waveform generating.
The frequency of square waveform is 5.3Khz \par
Duty cycle=65.14 percent \par

\begin{figure*}[h]
	\centering
	\includegraphics[scale=1]{sqwave.png}
	\caption{Square waveform }
	\label{FBD}
\end{figure*}


\subsection{PWM}
When we give input message signal at Pin-2 in Astable mode ,we get pwm output.we see when amplitude is high the duty cycle is high but when amplitude is low duty cycle is low.
\begin{figure*}[h]
	\centering
	\includegraphics[scale=1]{pwm output.png}
	\caption{PWM output}
	\label{FBD}
\end{figure*}




\subsection{PPM}
After generating of PWM we give this PWM in Monostable mode of 555 timer at pin-2.so we get output Pulse position modulation shown in below figure.
\par 
when their is falling edge in pwm signal we get ppm pulse at output.
\begin{figure}[h]
  \begin{subfigure}[b]{0.4\textwidth}
    \includegraphics[width=\textwidth]{ppm output.png}
    \caption{PPM OUTPUT-1}
    \label{fig:1}
  \end{subfigure}
  %
  \begin{subfigure}[b]{0.4\textwidth}
    \includegraphics[width=\textwidth]{ppm 2 output.png}
    \caption{PPM OUTPUT-2}
    \label{fig:2}
  \end{subfigure}
\end{figure}















\section{Conclusion/Sources of error}
 WE get First PWM moduluated signal after that we gave this PWM signal into Monostable mode of 555 timer so we get final PPM modulated signal..
 also when we calculate duty cycle theoretically by D=R1+R2/R1+2R2 we get Duty cycle 63.33 percent but practically we get duty cycle in astable mode of 555 timer is 65.14 percent.
 
 
\end{document}
